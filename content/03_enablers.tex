\section{Enablers}
\label{sec:enablers}

This section describes the enablers, which are the prerequisites that the innovating company brings
(or still has to fulfil) and that must match with the customer perspective that we introduced
in the previous section. Enablers can be based on the uniqueness of the company (i.e., its offering can not
easily be copied by competitors) or on operational excellence (i.e., the company is able to provide its
offering at a significantly lower cost than its competitors). The enablers are added to the company-side
of the \emph{Value Proposition Canvas} \citep{osterwalderValuePropositionDesign2014}.

\subsection{Uniqueness}

LinkedIn can provide a new service, i.e., \textit{Career Counseling as a Service} (CCaaS), based on its
unique position in the market, its broad access to career and job data, and access to state-of-the-art 
AI technology and machine learning algorithms.

\subsubsection*{Market Position}

LinkedIn is the world's largest professional network with more than 900 million members in over 200 countries
worldwide \citep{linkedinLinkedInPressromUs2023}. LinkedIn has a unique position in the market, see e.g.,
\cite{kaserAIpoweredCareerCounseling2023,99firmsLinkedInStatistics20232023}: 

\begin{itemize}
    \item LinkedIn's user base encompasses 900 million users (January 2023)
    \item LinkedIn is used by 49 million users weekly
    \item 365 million users have skills data on their profile (44\% of jobs filled with
        LinkedIn already use skills data as part of the recruiting)
    \item 50 million job searches per week (the widest reach in many Western countries)
\end{itemize}

\subsubsection*{Access to Data}

LinkedIn is one of very few companies that has access to the data of such a massive user base, including
very granular data on users' education, work experience, skills, and interests. LinkedIn also has access
to data on companies and their organizational structures, job advertisements, and hiring practices. This
data is a valuable resource for LinkedIn and can be used to train AI-based tools, e.g., job recommendation
systems, career path recommendation systems, and career counseling systems. Due to the combination of the
larger amount of data and more granular data, LinkedIn is in a unique position to train better AI algorithms
that its competitors. Another less obvious data advantage is the graph structure of the data in LinkedIn,
where users are connected to other users, to companies via jobs, and to schools via education. This graph
structure can be used to train graph-based machine learning algorithms that can provide better recommendations
than traditional machine learning algorithms due to overcoming data sparsity and cold start problems
\citep{zhangRecommendingGraphsComprehensive2023}.

\subsubsection*{Access to Technology and AI}

LinkedIn is part of Microsoft, one of the world's largest technology companies. Microsoft has invested heavily
in artificial intelligence (AI) and machine learning (ML) in recent years, including owning a stake in the hottest
of the AI companies, i.e., OpenAI \citep{openaiAnnouncementOpenAIMicrosoft2023}. Microsoft also owns GitHub and has
already proved that it can successfully integrate one of its companies with the offerings from OpenAI. In particular,
Microsoft and OpenAI have jointly developed GitHub Copilot, an AI-based code assistant that helps developers to write
better code, leveraging the huge database of GitHub and the AI know-how from OpenAI \citep{novetMicrosoftOpenAIHave2021}.
LinkedIn has similarly access to Microsoft's and OpenAI's AI and ML technologies. 

\subsection{Operational Excellence}

\subsubsection*{Low Cost}

\cite{floereckeSuccessFactorsSaaS2018} researched success factors for software-as-a-service (SaaS) business models. As 
main factor he found that \textit{``SaaS service[s] should be developed as a system comprising modular microservices
in order to meet the desired requirements in terms of cost advantages, performance and scalability''}. Accordingly, a
CCaaS business model should be designed as a system comprising modular microservices that are offered via an API layer.
LinkedIn can increase adoption by career counselors by offering a marketplace for career counselors that provides
no-code access to the CCaaS API layer. This allows career counselors to use the CCaaS services without the need 
of a full and costly technical integration of the API.

\subsubsection{Mastering Scale}

LinkedIn has successfully mastered scaling challenges in the past, including a phase of hypergrowth. As a result
LinkedIn has a proven track record in architecting scalable software systems based on a microservices approach
\citep{linkedinBriefHistoryScaling2015}.


\subsection{Gain Creators}

\subsection{Pain Relievers}

\subsection{Customer Centricity: Addressing Customer Needs}

\subsection{Need For Collaboration \& Co-creation}

LinkedIn is not specialized in career counseling but is a social network that connects professionals. It is not
LinkedIn's core business to provide career counseling. Therefore, LinkedIn needs to collaborate with specialized
companies that can provide these specialized career counseling services. Also, career counselors may work at different
locations around the world. For instance, our persona Sarah may wish to receive personal career coaching on-site at
her university in Zurich. LinkedIn is a digital platform that does not offer any physical presence points. Yet, LinkedIn
sits atop a mountain of valuable career data from its 900+ million members. This data can be used by LinkedIn to provide
highly specialized, machine learning based services. For instance, LinkedIn can provide a service that recommends career
counselors to its members based on their profile data. Career counselors can access machine learning based recommendations
of career paths for their clients using the CCaaS API layer. LinkedIn could design the CCaaS API layer in a way that career
counselors could actively or passively leave feedback on the job recommendations and career path recommendations provided
by LinkedIn, thereby helping LinkedIn to collect additional, refined data which are useful to further improve its machine
learning models. Both, LinkedIn and career counselors, thereby actively contribute to a higher value creation for the
clients that was not possible before. The CCaaS API layer presents a win-win(-win) situation for all three engaged parties:
LinkedIn, the counselors and the counseling clients.