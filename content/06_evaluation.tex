\section{Evaluation and Assessment}
\label{sec:evaluation}

In the following, we assess the business model and the contribution of the CCaaS innovation. Through a
multi-angled assessment, we can identify different types of contributions, risks and opportunities of CCaaS.
Risks may render the innovation useless for the customer, financially unviable, or even illegal. As part of
the assessment we can develop strategies to mitigate the identified risks, while being aware of the innovation's
advantages enables to exploit opportunities fully.

\subsection{Technical Feasibility and Considerations}

According to \cite{dornbergerDigitalInnovationDigital2021} technology can play two roles in the context of innovation
and transformation: new technologies can be enablers of innovation (``digital innovation''), or technology can be 
required to implement business or management ideas, such as digital ecosystems (``digital business transformation'').
In the case of CCaaS, we have technology as enabler, i.e., progress in computational power has fueled a renaissance
of sub-symbolic AI and deep learning in particular. On the other hand, technology---such as shared cloud computing
infrastructure---has brought costs down and makes it possible to viably offer new business models based on AI,
thereby enabling digital business transformation.

The CCaaS business model builds upon the availability of large amounts of user data and data science know-how.
The following technical aspects are particularly relevant in relation to career counseling as CCaaS offering:

\begin{itemize}
    \item \textbf{Data:} LinkedIn already has one of the largest---if not the largest---user database 
        on education, work experience, skills, companies and jobs. In particular, LinkedIn has collected
        verifiable information on the skills of its users: user can add skills to their profile and
        verify these skills by taking skill assessments quizzes \citep{kaserAIpoweredCareerCounseling2023}.
        Users that pass the quiz are awarded a skills badge that is displayed on their profile page.
    \item \textbf{Data Science:} The CCaaS platform will use AI models to provide
        recommendations for trainings and career paths. Through its parent company Microsoft,
        LinkedIn has access to computational resources and highly skilled data scientists and AI 
        researchers. Further, Microsoft owns a considerable stake in OpenAI, the creator of 
        ChatGPT, DALL-E, and other state-of-the-art AI models.
    \item \textbf{Cloud Computing:} Access to computational resources is a key requirement for training
        AI models and running them in production. Through its parent company Microsoft, LinkedIn has
        access to massive cloud computing resources, including GPUs from Microsoft Azure. Further,
        technical progress such as quantum computing may unleash manyfold more powerful computation 
        with much lower energy requirements in the near future.
\end{itemize}


\subsection{Economic Viability}

As a digital product, production costs for a ``unit'' of CCaaS (e.g., a single job recommendation) are
near zero. The main cost factor is an initial investment in infrastructure, data science know-how, and
training AI models. This is followed by ongoing costs for maintenance and further development of the AI
models and scaling of the infrastructure. Thus, the CCaaS business model is characterized by high fixed
costs and low variable costs. This means that the offering needs to reach a certain scale to break-even.
However, beyond this tipping point the business model is extremely profitable. In the following we try
to roughly estimate the fixed costs to calculate the break-even point based on the pricing scheme of the
API introduced previously.

We estimate the fixed costs of training each AI model to be in the range of 500,000 to 1,000,000 USD.
This includes the salaries for data scientists, the costs of building the data pipelines, and the costs
for computational resources (GPUs). Based on the system architecture there are 5 AI models to train,
which results in a total initial investment of 2.5 to 5 million USD. Further, we estimate the ongoing
costs for maintenance and further development of the AI models to be in the range of 200,000 to 500,000
USD per AI model and per year. This includes the salaries for data scientists and the costs for computational
resources. Thus, the total ongoing costs are in the range of 1 to 2.5 million USD per year. Using the higher
estimates and the average revenue of 1.075 USD per API request and 10.95 USD average revenue per customer
journey introduced previously, we can calculate the break-even point as follows:

\begin{itemize}
    \item Initial investment:
    \begin{itemize}
        \item 5 million USD
        \item 4.65 million API requests 
        \item 0.46 million counseling clients
    \end{itemize}
    
    \item Ongoing costs:
    \begin{itemize}
        \item 2.5 million USD per year
        \item 2.33 million API requests per year
        \item 0.23 million counseling clients per year
    \end{itemize}
\end{itemize}

To run the service profitably, the CCaaS platform needs to process at least 2.33 million API requests
or 230,000 counseling clients per year. Any requests surpassing this yearly threshold will help to offset
the initial investment costs up to 4.65 million API requests or 460,000 counseling clients. Beyond these
API requests and counseling clients, respectively, the CCaaS platform will be highly profitable due to 
low marginal costs. 230'000 counseling clients per year translates to under 5'000 clients per week.
Given LinkedIn's massive weekly user base of 49 million active users \citep{kaserAIpoweredCareerCounseling2023},
goal. This translates to a conversion rate of roughly 1 user out of 10'000 users that has to choose career
counseling services, which seems to be an attainable goal.

Finally, in terms of economic viability, we also need to assess the impact the new business model could 
have on the companies existing business model and revenue streams. When introducing a new business model 
a company has to take care not to disrupt itself and endanger existing business models and revenue streams.
The Transformative Business Stream Matrix introduced by \citet{schwafertsTransformativeBusinessStream2016}
allows companies to remain competitive by strategically developing additional business models that do not
put existing business models and revenue streams at immediate risk. Because LinkedIn is not engaged in 
career counseling, the CCaaS business model does not put any existing business models and revenue streams
at risk. Even more, through career counseling, LinkedIn's platform becomes more attractive for companies
to post their job ads. Thus, the CCaaS business model is a transformative business model that creates new
value and business streams while having a positive impact on existing business models and revenue streams.

\subsection{Competitive Advantage}

As introduced in Subsection \ref{subsec:uniqueness}, LinkedIn is in a unique position based on its massive 
and unparalleled user database in terms of size, granularity of data, and the graph-like structure. This user
database is the key asset and differentiator of LinkedIn versus the potential competitors. Further, LinkedIn's 
access to advanced AI technology via Microsoft and OpenAI is unparalleled. Only the few tech giants (Google,
Amazon, Facebook, Apple, Microsoft) have access to such resources. Thus, LinkedIn has a unique competitive
advantage in the combination of data and AI technology. None of the other tech giants has this unique 
combination that is required to build and run the CCaaS business model. Finally, we have seen that for
economic viability, the CCaaS business model requires a certain scale. LinkedIn is one of the companies that
has a user base that is large enough to run the CCaaS business model profitably. All of these factors 
are also extremely difficult and expensive to copy by potential competitors.

\subsection{Customer-Centricity}

We have previously introduced the customer perspective on career counseling in Section \ref{sec:customer_perspective}.
We extend here by embedding the innovation of AI-powered career counseling as a CCaaS API in the context of this 
customer perspective. The following aspects are relevant in this context:

\begin{itemize}
    \item \textbf{Reduced Costs:} Career counseling is currently not affordable to many. The CCaaS model
        will make career counseling accessible to a larger audience by reducing the costs of career
        counseling through shared infrastructure and shared AI models. CCaaS may fuel further innovation by 
        partners, such as e.g., making career counseling available online to people in remote areas.
    \item \textbf{Personalized Service:} Through the deployment of AI models as part of CCaaS, career counselors 
        will gain a better understanding of individual clients faster. This enables counselors to provide
        a more personalized service without increasing costs.
    \item \textbf{Assistive Technologies:} generative AI models can help job market entrants to write better CVs
        and application letters. This may currently typically not be demanded by clients due to the excessive costs.
        Generative AI can also be used as a chatbot interface to career counselors, thereby reducing the costs
        of the initial contact with a career counselor.
    \item \textbf{Avoiding Ethical Issues:} Ethical issues in AI-powered career counseling may include biased AI
        models, the problem of ``blackbox'' AI models that are not explainable, and Kafkaesque interactions between
        users and automated systems \citep{kaserAIpoweredCareerCounseling2023}. LinkedIn's CCaaS innovation will avoid 
        these issues by using a human-in-the-loop approach where specialized career counselors are the interface 
        to the client, while AI tools are used to assist the counselors in their work. The counselors can intervene
        whenever they deem it necessary and add reasoned explanations to the recommendations given to clients.
\end{itemize}


\subsection{Legal Considerations}

The following legal aspects are relevant in relation to career counseling and the CCaaS business model:

\begin{itemize}
    \item \textbf{Data Protection:} data protection is a key concern in career counseling as it is 
        centered on data about the education, work experience, skills, character traits, and personal 
        interests of the clients. Regulations such as the General Data Protection Regulation (GDPR) of
        the European Union \citep{europeanparliamentRegulationEU20162016} or the California Consumer
        Privacy Act (CCPA) \citep{californiastatelegislatureTitle81California2018} must be strictly followed
        in order to shield LinkedIn and the counselors from the risks of legal action. In the case of 
        CCaaS, we use data that has been provided by users to LinkedIn already. Further, users have to
        proactively opt in to career counseling services by using the career counselor matchmaking service:
        only counselors chosen by the client will have access to that client's data.
        Tighter regulations and differing legislations may mean that LinkedIn may have to offer regional
        versions of CCaaS that are operated in local data centers with a subset of the data from users of
        the same region (e.g., CCaaS Europe, CCaaS North America, etc.). This may affect the quality of 
        recommendations, but would affect competing services equally.
    \item \textbf{Liability:} the CCaaS platform provides recommendations for trainings and career
        paths. Because the CCaaS model does not promote full automation but rather a human-in-the-loop
        approach where specialized career counselors make the final recommendations, LinkedIn is not 
        liable for the outcome of the final recommendations given to users. However, the CCaaS platform
        should be liable for any damage caused to counselors based on downtimes of the CCaaS API. Downtimes 
        can be mitigated by well-known strategies for information system design such as redundancy, load
        balancing, and failover of traffic.
    \item \textbf{Intellectual Property:} the CCaaS platform will use AI models to provide
        recommendations for trainings and career paths. These AI models are based on data collected
        from the users as well as feedback data from counselors. There are currently open questions
        regarding who owns the intellectual property of such models trained on user data, see e.g.,
        \citep{anwerIPChallengesDatafueled2021}. LinkedIn should onboard counselors using contracts 
        that transfer all IP rights of the feedback data to LinkedIn.
\end{itemize}


\subsection{Sustainability Considerations}

Sustainability is a vast topic. To guide its analysis, the Sustainability Development Goals (SDGs) are a
good starting point. SDGs are a set of 17 goals defined by the United Nations in 2015 to achieve a better,
equitable and sustainable future by 2030 \citep{unitednationsTransformingOurWorld2015a}. The following SDGs
are of particular interest in relation to career counseling and the CCaaS business model:

\begin{itemize}
    \item \textbf{SDG Goal 4, Quality Education:} CCaaS decreases the cost of professional and personalized
        career counseling, including assessment of skills and recommendations for training, thereby
        contributing to a better education for all.
    \item \textbf{SDG Goal 5, Gender Equality:} with a proper design of the AI models, CCaaS can reduce 
        gender-related biases in recommendations of jobs, trainings and career paths.
    \item \textbf{SDG Goal 8, Decent Work and Economic Growth:} CCaaS can help people to find a job that
        matches their skills and interests. This can increase the productivity of the workforce and
        reduce the unemployment rate, while maximizing the income and satisfaction of the employees.
    \item \textbf{SDG Goal 10, Reduced Inequalities:} CCaaS can help to reduce inequalities by providing
        career counseling to people who previously could not afford it. Further, through proper design
        of the AI models, suitable trainings and career paths can be specifically tailored to the
        needs of underrepresented groups. However, career counseling is a service that is typically
        relevant for people who already have higher education. Widespread and affordable career
        counseling via CCaaS may hence contribute to widening the gap that is building between the
        highly educated workforce and the low-skilled workforce.
\end{itemize}


To summarize, CCaaS is a solution that aims to revolutionize the field of career counseling. By leveraging
user data, data science expertise, and cloud computing resources, CCaaS makes career counseling more affordable
and accessible to a broader audience. With the help of state-of-the-art AI models, career counselors can provide
personalized recommendations for trainings and career paths, improving the quality of their services without
increasing costs. Moreover, CCaaS addresses ethical concerns associated with AI in career counseling. While it
has the potential to reduce inequalities and cater to underrepresented groups, it's important to consider data
protection and intellectual property rights. However, the widespread availability of affordable career counseling
through CCaaS may contribute to the growing divide between the highly educated and the low-skilled workforce.
Finally, LinkedIn has a unique position to leverage on its database, access to AI expertise and scale to 
make CCaaS happen and not easily copied by competitors. We have also seen that CCaaS is an economically viable
business model.