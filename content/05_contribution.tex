\section{Contribution}
\label{sec:contribution}

\subsection{Business Idea}

\subsubsection{Features of CCaaS}



\subsection{Assessment of the Innovation}

In the following, we assess the contribution of the CCaaS innovation from four different angles, namely the
customer perspective, the technical perspective, the legal perspective, and the sustainability perspective.
Through a multi-angled assessment, we can identify different types of contributions, risks and opportunities.
Risks may render the innovation useless for the customer, financially unviable, or even illegal. As part of
the assessment we can develop strategies to mitigate the identified risks and better exploit opportunities.

\subsubsection{Customer Perspective}

We have previously introduced the customer perspective on career counseling in Section \ref{sec:customer_perspective}.
We extend here by embedding the innovation of AI-powered career counseling as a CCaaS API in the context of this 
customer perspective. The following aspects are relevant in this context:

\begin{itemize}
    \item \textbf{Reduced Costs:} Career counseling is currently not affordable to many. The CCaaS model
        will make career counseling accessible to a larger audience by reducing the costs of career
        counseling through shared infrastructure and shared AI models. CCaaS may fuel further innovation by 
        partners, such as e.g., making career counseling available online to people in remote areas.
    \item \textbf{Personalized Service:} Through the deployment of AI models as part of CCaaS, career counselors 
        will gain a better understanding of individual clients faster. This enables counselors to provide
        a more personalized service without increasing costs.
    \item \textbf{Assistive Technologies:} generative AI models can help job market entrants to write better CVs
        and application letters. This may currently typically not be demanded by clients due to the excessive costs.
        Generative AI can also be used as a chatbot interface to career counselors, thereby reducing the costs
        of the initial contact with a career counselor.
    \item \textbf{Avoiding Ethical Issues:} Ethical issues in AI-powered career counseling may include biased AI
        models, the problem of ``blackbox'' AI models that are not explainable, and Kafkaesque interactions between
        users and automated systems \citep{kaserAIpoweredCareerCounseling2023}. LinkedIn's CCaaS innovation will avoid 
        these issues by using a human-in-the-loop approach where specialized career counselors are the interface 
        to the client, while AI tools are used to assist the counselors in their work. The counselors can intervene
        whenever they deem it necessary and add reasoned explanations to the recommendations given to clients.
\end{itemize}


\subsubsection{Technical Perspective}

According to \cite{dornbergerDigitalInnovationDigital2021} technology can play two roles in the context of innovation
and transformation: new technologies can be enablers of innovation (``digital innovation''), or technology can be 
required to implement business or management ideas, such as digital ecosystems (``digital business transformation'').
In the case of CCaaS, we have technology as enabler, i.e., progress in computational power has fueled a renaissance
of sub-symbolic AI and deep learning in particular. On the other hand, technology---such as shared cloud computing
infrastructure---has brought costs down and makes it possible to viably offer new business models based on AI,
thereby enabling digital business transformation.

The CCaaS business model builds upon the availability of large amounts of user data and data science know-how.
The following technical aspects are particularly relevant in relation to career counseling as CCaaS offering:

\begin{itemize}
    \item \textbf{Data:} LinkedIn already has one of the largest---if not the largest---user database 
        on education, work experience, skills, companies and jobs. In particular, LinkedIn has collected
        verifiable information on the skills of its users: user can add skills to their profile and
        verify these skills by taking skill assessments quizzes \citep{kaserAIpoweredCareerCounseling2023}.
        Users that pass the quiz are awarded a skills badge that is displayed on their profile page.
    \item \textbf{Data Science:} The CCaaS platform will use AI models to provide
        recommendations for trainings and career paths. Through its parent company Microsoft,
        LinkedIn has access to computational resources and highly skilled data scientists and AI 
        researchers. Further, Microsoft owns a considerable stake in OpenAI, the creator of 
        ChatGPT, DALL-E, and other state-of-the-art AI models.
    \item \textbf{Cloud Computing:} Access to computational resources is a key requirement for training
        AI models and running them in production. Through its parent company Microsoft, LinkedIn has
        access to massive cloud computing resources, including GPUs from Microsoft Azure. Further,
        technical progress such as quantum computing may unleash manyfold more powerful computation 
        with much lower energy requirements in the near future.
\end{itemize}


\subsubsection{Legal Perspective}

The following legal aspects are relevant in relation to career counseling and the CCaaS business model:

\begin{itemize}
    \item \textbf{Data Protection:} data protection is a key concern in career counseling as it is 
        centered on data about the education, work experience, skills, character traits, and personal 
        interests of the clients. Regulations such as the General Data Protection Regulation (GDPR) of
        the European Union \citep{europeanparliamentRegulationEU20162016} or the California Consumer
        Privacy Act (CCPA) \citep{californiastatelegislatureTitle81California2018} must be strictly followed
        in order to shield LinkedIn and the counselors from the risks of legal action. In the case of 
        CCaaS, we use data that has been provided by users to LinkedIn already. Further, users have to
        proactively opt in to career counseling services by using the career counselor matchmaking service:
        only counselors chosen by the client will have access to that client's data.
        Tighter regulations and differing legislations may mean that LinkedIn may have to offer regional
        versions of CCaaS that are operated in local data centers with a subset of the data from users of
        the same region (e.g., CCaaS Europe, CCaaS North America, etc.). This may affect the quality of 
        recommendations, but would affect competing services equally.
    \item \textbf{Liability:} the CCaaS platform provides recommendations for trainings and career
        paths. Because the CCaaS model does not promote full automation but rather a human-in-the-loop
        approach where specialized career counselors make the final recommendations, LinkedIn is not 
        liable for the outcome of the final recommendations given to users. However, the CCaaS platform
        should be liable for any damage caused to counselors based on downtimes of the CCaaS API. Downtimes 
        can be mitigated by well-known strategies for information system design such as redundancy, load
        balancing, and failover of traffic.
    \item \textbf{Intellectual Property:} the CCaaS platform will use AI models to provide
        recommendations for trainings and career paths. These AI models are based on data collected
        from the users as well as feedback data from counselors. There are currently open questions
        regarding who owns the intellectual property of such models trained on user data, see e.g.,
        \citep{anwerIPChallengesDatafueled2021}. LinkedIn should onboard counselors using contracts 
        that transfer all IP rights of the feedback data to LinkedIn.
\end{itemize}


\subsubsection{Sustainability Perspective}

Sustainability is a vast topic. To guide its analysis, the Sustainability Development Goals (SDGs) are a
good starting point. SDGs are a set of 17 goals defined by the United Nations in 2015 to achieve a better,
equitable and sustainable future by 2030 \citep{unitednationsTransformingOurWorld2015a}. The following SDGs
are of particular interest in relation to career counseling and the CCaaS business model:

\begin{itemize}
    \item \textbf{Goal 4, Quality Education:} CCaaS decreases the cost of professional and personalized
        career counseling, including assessment of skills and recommendations for training, thereby
        contributing to a better education for all.
    \item \textbf{Goal 5, Gender Equality:} with a proper design of the AI models, CCaaS can reduce 
        gender-related biases in recommendations of jobs, trainings and career paths.
    \item \textbf{Goal 8, Decent Work and Economic Growth:} CCaaS can help people to find a job that
        matches their skills and interests. This can increase the productivity of the workforce and
        reduce the unemployment rate, while maximizing the income and satisfaction of the employees.
    \item \textbf{Goal 10, Reduced Inequalities:} CCaaS can help to reduce inequalities by providing
        career counseling to people who previously could not afford it. Further, through proper design
        of the AI models, suitable trainings and career paths can be specifically tailored to the
        needs of underrepresented groups. However, career counseling is a service that is typically
        relevant for people who already have higher education. Widespread and affordable career
        counseling via CCaaS may hence contribute to widening the gap that is building between the
        highly educated workforce and the low-skilled workforce.
\end{itemize}
