\section{Conclusion}
\label{sec:conclusion}

In this paper we have developed an innovative and transformative digital business model
for \textit{Career-Counseling-as-a-Service} (CCaaS). By leveraging the latest AI technologies,
such as recommender systems and generative AI models, that are embedded into an API layer,
we create a new digital ecosystem for career counseling. The CCaaS model is based on the
concept of value co-creation of different actors in the digital ecosystem and customer-centricity.
We have identified that career counseling services are not always affordable, and some counselors 
may lack the expertise required for a particular client. The AI-based CCaaS model addresses
these customer pains by providing better insights and enabling career counselors to automate
some of their processes. Further, new gains can be achieved by the CCaaS model, such as
24/7 availability due to automation of some services and better personalization of the
services as counselors can devote more time to advice the client instead of analyzing
the client's data.

LinkedIn is the central actor and enabler of the emerging digital ecosystem and acts as
the orchestrator of the ecosystem. It has a unique position due to access to vast amounts 
of highly granular, graph-like data on users' employments, education, and skills. Further,
LinkedIn is part of Microsoft, which also owns a stake in OpenAI. This allows LinkedIn
to leverage the latest AI technologies and provide them to the ecosystem actors via
their AI models and the CCaaS API layer. We have identified that the CCaaS model
is a viable business model for LinkedIn as it can leverage its unique position and
data to create a new digital ecosystem and generate new revenue streams without
endangering its existing business model and revenue streams.

Technically, CCaaS is designed as a set of microservices that are deployed as an 
API layer. The API layer is the central component of the CCaaS model as it enables
the integration of the different actors in the ecosystem. The microservices' architecture
allows for a flexible and scalable deployment of the CCaaS model, including evolving
the offering over time by adding or retiring microservices. The CCaaS model is
designed to be deployed on the cloud platform of Microsoft Azure (the parent company)
to leverage scalability aspects.

All in all, the CCaaS model represents a unique opportunity for LinkedIn to
enter into a business model and create a new revenue stream.
