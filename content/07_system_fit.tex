\section{System-FIT}
\label{sec:system_fit}

System-FIT is a framework introduced by \citet{schwafertsDigitalBusinessDevelopment2020} and helps
companies to evaluate the fit of their business model with an existing or emerging digital ecosystem.
In particular, \cite{schwafertsDigitalBusinessDevelopment2020} argues that most companies are not able
to create their own digital ecosystem and therefore have to join an existing one. The framework
thus helps companies navigate, evaluate and choose the right digital ecosystem to join. 
In the case of career counseling, we have seen in Subsection \ref{subsec:ecosystem} that it
does not meet the definition of a true ecosystem yet. LinkedIn can introduce the CCaaS model to 
create a new digital ecosystem for career counseling as a value co-creation network with LinkedIn
as the most powerful company at its center. In the following we will thus evaluate the fit of CCaaS
as a transformative business model in the emerging digital ecosystem of AI-based career counseling.


\subsection{Fit of Uniqueness}

LinkedIn is the most powerful company in the emerging digital ecosystem of AI-based career counseling
due to its user data and easy access to AI technology. It is needed as the central actor and provider
of the CCaaS API platform. It is unlikely that any other company will be able to create this ecosystem.
The need of LinkedIn is thus absolutely essential.
Similarly, the digital ecosystem that will be created is unique for the Western world. While LinkedIn
is blocked or marginalized in some countries (e.g., fully blocked in Russia, and marginalized in China)
we may see parallel ecosystems emerge in these countries. However, it is unlikely that these ecosystems
will spread out into the LinkedIn core markets, such as North and South America or Europe.
The alternatives to CCaaS are traditional career counseling services, which are either not digital
or scattered. We do not see another, competing digital ecosystem emerging which would be equal in
terms of AI capabilities and user data. The uniqueness of CCaaS is thus very high and LinkedIn as its
central actor of the emerging ecosystem is absolutely essential.


\subsection{Fit of Management}

LinkedIn is itself the most powerful company in the emerging digital ecosystem of AI-based career
counseling. It thus does not need to adapt to any other, more powerful company. However, the management
of LinkedIn should consider the other actors in the digital ecosystem and their needs. In particular,
the needs and pains of career counselors and clients need to be integrated into the decision-making
processes. As argued previously, the CCaaS model relies on value co-creation and thus requires strong 
partners that are enabled to innovate via the CCaaS API layer. This form of open innovation will also 
benefit LinkedIn through increased usage of the CCaaS API that will lead to higher revenues and more
feedback data. LinkedIn thus has to act as the orchestrator of the emerging digital ecosystem and
ensure proper governance (such as same access and pricing conditions for all players). This will build
multilateral trust, lead to alignment of the ecosystem actors and ultimately facilitate and accelerate
the joint value-creation. With a steadily increasing workforce with tertiary education in the Western
world, the demand for counseling services will continue to grow. The digital ecosystem will thus 
provide plenty of opportunities for many actors to co-create value and grow together.


\subsection{Fit of Structure}

The (infra-)structure of the digital ecosystem does not exist yet as such. However, LinkedIn has the
data and access to the computation and AI technology resources via its parent company Microsoft. As
introduced in subsection \ref{subsec:system_architecture}, the system architecture of CCaaS is based 
on a microservice architecture. This allows for a modular and flexible system structure that can be
easily adapted to the needs of the ecosystem actors as well as to evolving needs: new microservices 
may easily be added while microservices that are no longer needed could be retired. Of course---referring 
back to Fit of Management---LinkedIn has to be cautious in taking decisions to retire microservices 
and consider the needs of the ecosystem actors. The CCaaS API layer should be kept as stable as possible
and backward-compatible to ensure that the ecosystem actors can rely on it. Any changes to the API 
or removing of services have to planned in advance and communicated to the ecosystem actors to give
them enough time to adapt. Ideally the CCaaS API layer will be extended over time to provide more
functionality and thus enable the ecosystem actors to innovate and create more value. Through the 
API-based architecture, other actors in the ecosystem are empowered to customize their own offerings.


\subsection{Fit of Partnering}

• Can I be integrated in Co-Creation ? \\
• Do I share the value and ethic that that prevail in this network ? \\
• Do we have compatible expectations about performance ? \\
• Can there arise a competition for power ? Could I win this ? \\
• Can you achieve a situation of symbiosis ? \\
• Do the partners have a choice ? \\
• Can anyone steal my innovation ? \\
• Who takes the largest share of the profit ? \\

\subsection{Fit of Customer Understanding}

• Do I know and share to customer expectation of the most powerful company ? \\
• Do I know and share the value and ethic that stands behind this customer expectation ? \\
• Do I meet the expectations of the customer and the most powerful company in terms of quality and branding ? \\
