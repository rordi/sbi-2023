\section{Introduction}
\label{sec:introduction}

Latest developments in generative AI have unleashed a new wave of speculations on how industries are going to
evolve over the next few years, see, e.g., \cite{chuiHowGenerativeAI2022}. Many companies are reconsidering how
AI in general and generative AI in particular will affect their industries and ecosystems. Once such industry is
career counseling, which is also known as career guidance. Career counseling is the discipline and set of services
related to designing career paths and consulting individuals regarding their career opportunities.

In this paper we explore a new innovative business model in career counseling, namely AI-powered career counseling
embedded in a social and digital ecosystem of career counseling. We will first describe the customer perspective of
this business model using the Value Proposition Canvas \citep{osterwalderValuePropositionDesign2014}. In particular,
we will look at  the customer perspective in terms of possible customer segments and their respective needs
(\textit{gains} and \textit{pains}) in Section \ref{sec:customer_perspective}. Further, we will describe the drivers
and enablers of this new business model. Drivers encompass societal, technological and environmental trends and
developments that make this business model possible and are described in Section \ref{sec:drivers}. Enablers encompass
the resources available to the innovating company thereby increasing the likelihood of realization and viability of the
new business model, and are described in Section \ref{sec:enablers}. Then, we will describe the business model itself
using the Business Model Canvas \citep{osterwalderBusinessModelGeneration2010} in Section \ref{sec:business_model}.
In particular, we will look at the value proposition, customer segments, channels, customer relationships, key
resources, key activities, key partnerships, revenue streams, and cost structure of this business model.
Further, we will detail the specific contribution of (strategic) innovation in this business model in Section
\ref{sec:contribution}. We will then evaluate the business model in terms of its viability and feasibility in Section
\ref{sec:evaluation}. Finally, we will describe the fit of this business model with the system in which it is embedded
in Section \ref{sec:system_fit}. We will conclude with a summary of our findings in Section \ref{sec:conclusion}.

In the remainder of this section, we will give a background on career counseling as well as the strategic innovation
potential that stems from the latest generation of AI technologies.

\subsection{Career Counseling}

Specifically, the services
in career counseling include:

\begin{itemize}
    \item Career Assessment: Assessment of the traits of the client, including identifying their preferences,
            strengths, skills, and values and matching those with suitable career paths.
    \item Development \& Training
    \item Job Search Assistance
    \item Career Transitions
    \item Entrepreuneurship-related services
\end{itemize}

While LinkedIn is arguably the most dominant player in terms of employee data in Western countries,
there are plenty of other companies that have access to employee data. However, the databases of competitors are
not as large as LinkedIn's or are focused on a particular country or regions. While these databases are certainly
relevant in some career paths and countries, they are not as relevant in others or for international careers.

% @TODO fact check the whole paragraph - if this is all true, this can be an interesting addition!
In Germany, there is Xing, which is a German company that is also active in Switzerland and Austria.
In China, there is Maimai, which is
a Chinese company that is also active in China. In India, there is Naukri, which is an Indian company that is also
active in India. In Russia, there is HeadHunter, which is a Russian company that is also active in Russia. In Japan,
there is Wantedly, which is a Japanese company that is also active in Japan. In South Korea, there is Saramin, which
is a South Korean company that is also active in South Korea. In Brazil, there is Vagas, which is a Brazilian company
that is also active in Brazil. In Mexico, there is OCC, which is a Mexican company that is also active in Mexico.
In the United States, there is Indeed, which is an American company that is also active in the United States. In
Canada, there is Workopolis, which is a Canadian company that is also active in Canada. In Australia, there is
Seek, which is an Australian company that is also active in Australia. In New Zealand, there is Trade Me, which is
a New Zealand company that is also active in New Zealand. In South Africa, there is CareerJunction, which is a
South African company that is also active in South Africa. In Nigeria, there is Jobberman, which is a Nigerian
company that is also active in Nigeria.

\subsection{Potential for AI Tools in Career Counseling}